\documentclass[14pt]{extarticle}
\usepackage[english]{babel}
\usepackage[utf8x]{inputenc}
\usepackage[T1]{fontenc}
\usepackage{scribe}
\usepackage{listings}
\usepackage{relsize}

\Scribe{Aditya Diwakar}
\Lecturer{Frederic Faulkner}
\LectureNumber{1}
\LectureDate{January 11, 2022}
\LectureTitle{Welcome \& Big-O}

\lstset{style=mystyle}
\setlength{\parindent}{0pt}

\begin{document}
	\MakeScribeTop

    Welcome to CS3510! Lecture began by reviewing the syllabus, please read
    this syllabus and become familiar with the class structure.

    \section{Big O}
    Why do we care about Big O? It's hard to talk about performance because
    runtime (by seconds) varies by computer by computer. Time based performance
    is affected by a variety of things: code efficiency, hardware, background
    tasks, etc. We only care about one item: \textit{code efficiency}.\\

    We want to ignore multiplicative (hardware performance) and additive (i.e.
    loading time) constants. Hence, we introduce Big-O. We say the following:

    \begin{center}
        $f(n)$ is $O\left(g(n)\right)$ if $\exists c, k: f(n) \leq c\cdot g(n)$
        when $n > k$ (for large enough inputs).
    \end{center}
    Another way to say this is that if $f(n) = O\left(g(n)\right)$, then we can
    say that the growth of $f(n)$ is bounded above by $g(n)$.\\

    The "O" can be thought of equivelantly as $\leq$. When we say $f(n)$ is
    $O\left(g(n)\right)$, this is somewhat related to $f(n) \leq g(n)$. We
    are comparing the rates of growth (rather than the functions directly).
    We also have:
    \begin{center}
        $\Omega$: $f(n)$ is $\Omega\left(g(n)\right)$ if $g(n)$ is
        $O\left(f(n)\right)$ (reverse)\\[2mm]
        $\Theta$: $f(n)$ is $\Theta\left(g(n)\right)$ if $f(n)$ is
        $O\left(g(n)\right)$ AND $g(n)$ is $O\left(f(n)\right)$ (both ways)
    \end{center}

    \pagebreak
    \subsection{Examples}
    \begin{enumerate}
        \item $n$ is $O(n^2)$ ($n^2$ grows faster than $n$)
        \item $n^2$ is $\Omega(n)$ ($n$ grows slower than $n^2$)
        \item $n$ is $\Theta(3n + 27001)$ (ignoring constants, these grow at
            the same rate)
    \end{enumerate}

    Proving (3) using the definition from above:
    \begin{align}
        \frac{n}{3n + 27001} < 1\quad\quad n\geq 0
    \end{align}
    \label{fwdex1}
    and the reverse direction...
    \begin{align}
        \frac{3n + 27001}{n} = \frac{3n}{n} + \frac{27001}{n} 
        = 3 + \underbrace{\frac{27001}{n}}_{\leq 27001 \text{ when } n \geq 1}
        \leq 3 + 27001 = 27004
    \end{align}
    \label{fwdex2}
    Hence, using equation (1), we have shown that $n$ is $O(3n + 27001)$
    using (2), $3n + 27001$ is $O(n)$, hence $n$ is $\Theta(3n + 27001)$.
    \hfill
    \square

    \subsection{Important Notes of Big O}
    \begin{enumerate}
        \item Ignore multiplicative constants 
        \item $n^a$ dominates $n^b$ if $a > b$: $n^2$ is $O(n^4)$ and
            $n^4$ is NOT $O(n^2)$ so we can focus on what contributes to
            growth the most, simplifying polynomials.
        \item Exponentials dominate any polynomial, for example:
            \begin{align*}
                2^{0.5n} &\text{ is } \Omega(n^{16,000} + n^{10,000,000})\\
                a^n &\text{ is dominated by } b^n \text{ for } a < b
            \end{align*}
        \item Any polynomial dominates any log
    \end{enumerate}
    For practice, let us compare $2^n$ vs $2^{n + 1}$ by using a ratio method:
    \begin{align*}
        \frac{2^n}{2^{n+1}} = \frac{1}{2}
        \quad
        \text{and}
        \quad
        \frac{2^{n+1}}{2^n} = 2
    \end{align*}
    hence $2^n$ is $O\left(2^{n+1}\right)$ and $2^{n+1}$ is 
    $O\left(2^n\right)$ meaning $2^{n+1}$ is $\Theta(2^n)$.

    \pagebreak
    \subsection{More Examples}
    \begin{enumerate}
        \item What is the relationship between $n$ and $3^{\log_5 n}$?\\[2mm]
            Using logarithm rules, notice that $3 = 5^{\log_5 3}$ so
            we can rewrite $3^{\log_5 n}$:
            \begin{align*}
                3^{\log_5 n} 
                = \left(5^{\log_5 3}\right)^{\log_5 n}
                = 5^{\left(\log_5 3 \cdot \log_5 n\right)}
                = \left(5^{\left(\log_5 n\right)}\right)^{\log_5 3}
                = \boxed{n^{\log_5 3}}
            \end{align*}
        and since $3 < 5$, then $\log_5 3 < 1$, meaning $n$ dominates
        $3^{\log_5 n}$.
        
        \item What is the relationship between $\sqrt{n}$ and $\log^2 n$?\\[2mm]
            Recall that $\sqrt{n} = n^{0.5}$ meaning $n^{0.5}$ is polynomial
            and using rule (4), we know that this dominates any logarithm 
            hence $\log^2 n$ is $O\left(\sqrt{n}\right)$

        \item What is the relationship between $2^n$ and $2^{n/2}$?
            \begin{align*}
                2^{n/2} = 2^{0.5n} = \left(2^{0.5}\right)^n = 
                \left(\sqrt{2}\right)^n
            \end{align*}
            This means that $2^{n/2}$ has a smaller base, and therefore is
            $O\left(2^n\right)$.
    \end{enumerate}
\end{document}
